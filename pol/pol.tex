

\chapter{Pol}
Em reuniões em nov/2014 com Moser, Bednarski e Alex, ficou decidido:
\begin{itemize}
    \item Cada estrela (alvo ou padrão) terá um PREFIXO único de observação.
Essa lista é gerada da seguinte forma: copia-se a coluna no arquivo \textit{
planejamento\_LNA.xls} num arquivo de texto, que será lido pelas rotinas
(arquivo pyhdust/pol/alvos.txt). Destaca-se que a identificação do alvo é feita
desta forma, não importando o prefixo dos arquivos fits. 
    \item A estrutura de diretórios ``local'' é \textit{noite/prefixoalvo\_n}.
Assim, todo sub-diretório da noite é considerado um alvo, exceto \textit{calib}. 
Caso uma estrela tenha sido observada em mais de uma sequencia, adicionar
\textit{\_n}, onde pode ser um número (e.g., 2) ou um descritivo (e.g., $a2$).
Tudo o que está em subdiretórios dos alvos é ignorado.
    \item A estrutura de diretórios será \textit{dadospuros/noite/alvo} e
\textit{reduzidos/noite/alvo}. O script \textit{pur2red} limpa o resultado da
redução dos dados puros. O script \textit{red2pur} copia os resultados da
redução em reduzidos para a pasta dos dados puros.
    \item O critério para a escolha do(s) arquivos \textit{out(s)} será o de
minimizar os erros em blocos de 16 ou 8+8 posições, mantendo o máximo de pontos
independentes possível. Os detalhes estarão no manual das rotinas.
%o
%seguinte: (a) se somente 8 posições, menor \textit{sigma}; (b) Se $8<n<24$,
%menor erro entre sequências de 8 e a de maior $n\leq16$; (c) se bloco for
%múltiplo de 16, usa blocos de 16 ou 8+8 de menor erro. Se for múltiplo de 8,
%usa blocos de 16 ou 8+8, iniciando ou terminando com 8. (d) Se não for
%últiplo de 8 ou 16, desloca o início da sequência a fim de minimizar o erro.
    \item As rotinas \textsc{python} propostas são as seguintes: (i)
\textit{genStdLog}, (ii) \textit{genObjLog}, (iii) \textit{genStdDat}, 
(iv) \textit{genTarget}, (v) \textit{pur2red}, (vi) \textit{red2pur} e (vii)
\textit{listNights}. Consultar o manual das rotinas para a lista completa e
detalhes.
    \item É possivel demonstrar que $P=\sqrt{Q^2+P^2}$ e $\sigma_P=\sigma_{QU}$.
\end{itemize}

Passos da redução: 
\begin{enumerate}
    \item Obtem-se a lista de noites do objeto (ou via página Beacon, ou via
rotina \textit{listNights} caso tenha acesso a pasta \textit{puros}).
    \item Gera-se os arquivos de calibração-padrão (rotina \textsc{iraf}
\textit{calib}).
    \item Reduz-se as padrões da noite no \textsc{iraf}. Roda-se 
\textit{genStdLog}. A rotina imprime alertas
de dados de padrões não reduzidos. Colunas da saída de \textit{genStdLog}:
MJD|target|filt|calc|out|flags| (um \textit{out} por filtro/padrão). 
    \item Reduz-se os alvos da noite (\textsc{iraf}). Roda-se \textit{genObjLog}
 e imprime alertas de dados não reduzidos/estrelas fora da nomenclatura.
Colunas da saída de \textit{genObjLog}: MJD|target|filt|calc|out\_n|flags| (pode
 haver mais de um \textit{out} por filtro/alvo; registro em nova linha). 
    \item Roda-se \textit{genStdDat}.
A rotina faz: (i) olha na saída de \textit{genObjLog} os filtros/calcitas dos 
alvos utilizados, (ii) e espera encontrar as padrões
correspondentes; (iii) caso não encontre uma padrão, faz interpolação (de
filtros). Se faltou para toda a calcita, pede para o usuário fazer a
 referência a saída de outro \textit{genStdLog}.

    \item Reduz-se todas as noites (de interesse). Roda-se
\textit{genTarget} para um, múltiplos ou todos os alvos. Procura noite à noite
o alvo na saída de \textit{genObjLog}, calibrado com \textit{genStdDat} da
noite correspondente (detalhe: nesta configuração, os arquivos \textit{outs}
serão lindos neste momento. Ou seja, o usuário depois dos passos 1, 2 e 3, ainda
 precisa ter acesso a estes arquivos). Neste momento, o usuário confirma qual
ângulo de
referência da padrão será utilizado (padrão é o ``publicado''). Caso exista
mais de uma padrão para o filtro/calcita, faz uma média do $\Delta\theta$.
Colunas da saída de \textit{genTarget} (saída individual para cada estrela):
\newline MJD|noite|filt|calc|ang.ref|del.th|P|Q|U|th|sigP|sigth|flags|.
\end{enumerate}
    
Outras definições sugeridas:
\begin{itemize}
    \item No caso de um alvo observado com duas calcitas (como padrões), não
colocar no mesmo diretório. Isso complica muito o desflexibiliza muito as
rotinas que gerenciam os artivos \textit{*.out}. A regra é criar uma nova pasta
com o sufixo ``\_a2'' contendo os arquivos fits. A raiz destes arquivos é
arbitrária.
    \item Além disso, outra situação é necessária para se abrir outra pasta:
ao se iniciar uma nova sequência temporal.
    \item Nomenclatura de observação \textit{target\_a0\_g1\_f.fits} onde $a0$
e $g1$ são argumentos opcionais (calcita e ganho/config$.$ do CCD) e $f$ é o
filtro. Na ausência destes parâmetros, assume-se valor padrão (ou lê-se no
\textit{header} dos fits).
    \item Os avisos em tela \textit{Warning} avisa o usuário que uma informação
foi registrada, mas ela não é a otimizada. Por exemplo, na ausência de uma das
versões de redução (.1 ou .2) ou do agrupamento de posições da lâmina (08 ou
16).
    \item Os avisos em tela \textit{Error} avisa o usuário que uma informação
foi perdida. Dependendo o erro, o script será interrompido (não gerando nenhum
output parcial).
\end{itemize}

Definições das flags.
\begin{itemize}
    \item[0] As flags são cumulativas. Se estiver 0, significa que tudo ocorreu
bem (nenhuma decisão ou aproximação no processo de redução foi feita).
Se ocorreram, terá uma ou mais das flags abaixo.
    \item[pA] Estrela sem padrão para calibração de $dth$ (output é salvo mesmo
assim).
    \item[pB] A correção $dth$ foi feita usando uma estrela de outra noite.
    \item[pC] A correção $dth$ foi feita usando duas ou mais estrelas.
\end{itemize}

